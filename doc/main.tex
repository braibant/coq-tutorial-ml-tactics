\documentclass[a4]{article}
\usepackage{color}
\usepackage{listings}
\usepackage{lstocaml}
\usepackage{lstcoq}

\newenvironment{twolistings}%
{\noindent\begin{tabular*}{\linewidth}{@{}c@{\extracolsep{\fill}}c@{}}}%
{\end{tabular*}}

\begin{document}
\begin{abstract}
  We try to build a tutorial on how to write an OCaml tactic for Coq,
  with a focus on reification tactics.
\end{abstract}

\section*{Introduction}
The long term goal of this tutorial is to reify the equation on the
left, into a reified version on the right, that exposes the syntax
tree of the expressions.

\begin{twolistings}
  \begin{coq}


a,b: nat
========= 
a + S b + 2 = a + b + 3
  \end{coq}
&
  \begin{coq}
a,b: nat
left := (a_plus (a_plus (a_const a) (a_succ (a_const b))) (a_const 2))
right := (a_plus (a_plus (a_const a) (a_const b)) (a_const 3))
========= 
left  = right
  \end{coq}

\end{twolistings}
\section{OCaml representation of a Coq term}
The first step to get 
\begin{coq}
  
\end{coq}
\section{A first tactic}
\section{Reification in Coq}

\appendix
\section{The makefile}
\section{A walkthrough of Coq sources}

\end{document}

%%% Local Variables: 
%%% mode: latex
%%% TeX-master: t
%%% End: 
